% !TEX root = SystemTemplate.tex

\chapter{Experimental Log}

A log of all research activities.   

%% If you have multiple labs, you may want to break the labs into sections, check 
%% with the profession on format.

\section{Benchmarking the Individual Computers}

\begin{description}
\item [9/17/15]  PcDuino isn't working according to Dr. Karlsson. The PcDuinos are about \$160 each, so chances were that the PcDuino wasn't going to be selected for the cluster. We will not put the PcDuino in consideration with our benchmarking.
\item [9/17/15]  PcDuino isn't working according to Dr. Karlsson. The PcDuinos are about \$160 each, so chances were that the PcDuino wasn't going to be selected for the cluster. We will not put the PcDuino in consideration with our benchmarking.
\item [9/22/15] We begin work on benchmarking the remaining candidates. The code we will be using to benchmark the two devices will test the addition, multiplication, division, trigonmetric function in single and double point precision of two massively large arrays filled with random numbers.
\item [9/29/15] OpenMP is added to the benchmark code so the program runs on all cores. Results are as follows: \newline

\begin{center}
\begin{tabular}{ | l || l | l | l | l | }
\hline
\multicolumn{5}
{ |c| }{ Length of Time (seconds) } \\
\hline
Device & Addition & Multiplication & Division & Sine \\
\hline
ODroid 4xU & 29.925 & 31.341 & 37.032 & 227.40 \\
\hline
Raspberry Pi 2B & 221.645 & 221.034 & 297.204 & 1468.63 \\
\hline
\end{tabular}
\end{center}

\item [9/30/15] The gigaflops are calculated. \newline
\begin{center}
\begin{tabular}{ | l || l | l | l | l | }
\hline
\multicolumn{5}
{ |c| }{ Gigaflops } \\
\hline
Device & Addition & Multiplication & Division & Sine \\
\hline
ODroid 4xU & 0.311 & 0.297 & 0.251 & 0.0410 \\
\hline
Raspberry Pi 2B & 0.0420 & 0.0421 & 0.0313 & 0.00634 \\
\hline
\end{tabular}
\end{center}

\item [10/1/15] The wattage is measured when the devices are running these operations. Using the wattage, the metric of GFlops/Dollar/Watt is calculated. \newline

\begin{center}
\begin{tabular}{ | l || l | l | l | l | }
\hline
\multicolumn{5}
{ |c| }{ Gigaflops per Dollar per Watts } \\
\hline
Device & Addition & Multiplication & Division & Sine \\
\hline
ODroid 4xU & 0.00028 & 0.000268 & 0.000226 & 0.0000369 \\
\hline
Raspberry Pi 2B & 0.0003 & 0.0003 & 0.000224 & 0.0000453 \\
\hline
\end{tabular}
\end{center}

\item [10/1/15] The results show that the Raspberry Pi and the ODroid perform nearly the same. The Raspberry Pi in our benchmarking proved the best. However, it is inconclusive as to which computer will be used.

\end{description}