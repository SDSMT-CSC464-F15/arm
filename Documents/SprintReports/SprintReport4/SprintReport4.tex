\documentclass{article}
\usepackage{fancyhdr}
\usepackage{color}
\usepackage{titlesec}
\usepackage[margin=0.5in]{geometry}
\usepackage{tabto}
\usepackage{tabularx}

\definecolor{TitleColor}{rgb}{0.0, 0.0, 0.0}
\definecolor{SectionColor}{rgb}{0.3, 0.3, 0.3}
\definecolor{SubsectionColor}{rgb}{0.4, 0.4, 0.4}

\titleformat{\section}{\normalfont\Large\bfseries\color{SectionColor}}{\thesection}{1em}{}[{\titlerule[2.0pt]}]
\titleformat{\subsection}{\normalfont\bfseries\color{SubsectionColor}}{\thesection}{1em}{}

\title{\normalfont\Large\bfseries\color{TitleColor}Sprint Report \#4}
\date{\normalfont\bfseries\color{TitleColor}\today}

\begin{document}

\maketitle

\section*{Team Overview}
\subsection*{Project}
\tab{ARM Cluster}

\subsection*{Members}
\begin{itemize}
	\item Andrew Hoover
	\item Christine Sorensen
\end{itemize}

\subsection*{Sponsor}
\tab{Dr. Christer Karlsson}

\subsection*{Meeting Times}
\tab{Tuesdays and Thursdays at 1:00pm}
\subsection*{Work Times}
\tab{Tuesdays and Thursdays at 10:00am}

\section*{Sprint Overview}
\subsection*{Work for this sprint included:}
\begin{itemize}
	\item Graph Benchmark Results
	\begin{itemize}
		\item Ran LINPACK on a one to eight devices and recorded results.
		\item Graphed the speeds using Python libraries.
	\end{itemize}
	\item Compare Cluster to i7
	\begin{itemize}
		\item Installed LINPACK on Dr. Karlsson's i7 named Red Queen
		\item Ran the test and recorded the gigaflops on four to eight cores.
	\end{itemize}
	\item Created LINPACK as debian package for Arm.
	\item Researched USB communcation.
	\begin{itemize}
		\item No method for USB communication was found for USB 3.0.
		\item USB 2.0 was determined to be too slow to be feasible.
	\end{itemize}
	\item Researched GPIO communication.
	\begin{itemize}
		\item Communication by using the file system in /sys/class/gpio was demonstrated to work.
		\item WiringPi for ODroid was installed.
		\item The kernels on the devices were updated to be able to use WiringPi.
		\item Communication in C using WiringPi and the GPIO pins was demonstrated to work.
	\end{itemize} 
	\item MICS conference.
	\begin{itemize}
		\item Wrote the abstract for our research to deliver to MICS.
		\item Reviewed the abstract with our client.
		\item Peer reviewed the abstract with antoher local team attending MICS.
		\item Submitted the abstract to MICS.
	\end{itemize}
\end{itemize}

\section*{Deliverables}
\begin{itemize}
	\item Graphs of total gigaflops performed depending on amount of devices used.
	\item Debian package of LINPACK for ARM.
	\item Found USB communication to not be feasible.
	\item Able to send bits over GPIO between ODroid devices.
	\item MICS abstract.
\end{itemize}

\section*{Activities}

\subsection*{Andrew Hoover}
\begin{itemize}
	\item Created LINPACK debian package.
	\item Ran LINPACK on differing amount of devices in the cluster and saved the results.
	\item Installed LINPACK on Dr. Karlsson's i7 to compare to the cluster.
	\item Researched USB communication.
	\item Debugged WiringPi.
	\item Spent some more time debugging WiringPi.
	\item Was able to get WiringPi to work for C.
	\item Updated kernel's of ODroids.
	\item Edited sprint report.
\end{itemize}
\subsection*{Christine Sorensen}
\begin{itemize}
	\item Wrote MICS abstract.
	\item Wrote Python code to graph LINPACK results.
	\item Created documentation.
	\item Researched GPIO communication.
	\item Talked to faculty about GPIO and instuctional uses for the cluster.
	\item Debugged WiringPi.
	\item Spent some more time debugging WiringPi.
	\item Was able to get WiringPi to work for C.
	\item Wrote initial sprint report.
\end{itemize}

\subsection*{Work that is carried over into Sprint 5 is as follows:}
\begin{itemize}
	\item Use protocols for data transfer over GPIO.
	\item Benchmark those protocols and compare to Ethernet.
	\item Continue working on MICS.
	\item Write abstract for SDSMT's Research Symposium.
\end{itemize}

\section*{Backlog}
\begin{itemize}
	\item MICS presentation.
	\item SDSMT Research Symnposium. 
	\item Design Documentation.
	\item Design Fair.
\end{itemize}
\end{document}
