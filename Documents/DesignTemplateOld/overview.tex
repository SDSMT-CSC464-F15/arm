% !TEX root = SystemTemplate.tex

\chapter{Overview and concept of operations}

%%*The overview should take the form of an executive summary.  Give the reader a feel 
%%for the purpose of the document, what is contained in the document, and an idea 
%%of the purpose for the system or product. 


\section{Scope}
%What scope does this document cover? 

\section{Deliverables}
\begin{itemize}
	\item ARM Cluster
	\item Research Symposium
	\item Design Fair
	\item Documentation
\end{itemize}

\section{Purpose}
%The purpose of this project is to build a cluster of 6-12 single-board computers. This cluster %should perform floating-point operations. With research, the cluster will have a new mode of %communication and be able.

\subsection{Major System Component \#1}
%Describe briefly the role this major component plays in this system. 

\subsection{Major System Component \#2}
%Describe briefly the role this major component plays in this system. 

\subsection{Major System Component \#3}
%Describe briefly the role this major component plays in this system. 

\section{Systems Goals}
%Briefly describe the overall goals this system plans to achieve.
%These goals are typically provided by the stakeholders.  This is not
%intended to be a detailed requirements listing.  Keep in mind that
%this section is still part of the Overview.

\section{System Overview and Diagram}
%Provide a more detailed description of the major system components
%without getting too detailed.  This section should contain a
%high-level block and/or flow diagram of the system highlighting the
%major components.  See Figure~\ref{systemdiagram}.  This is a floating
%figure environment.  \LaTeX\ will try to put it close to where it was
%typeset but will not allow the figure to be split if moving it can not
%happen.  Figures, tables, algorithms and many other floating
%environments are automatically numbered and placed in the appropriate
%type of table of contents.  You can move these and the numbers will
%update correctly.

\section{Technologies Overview}
%This section should contain a list of specific technologies used to
%develop the system.  The list should contain the name of the
%technology, brief description, link to reference material for further
%understanding, and briefly how/where/why it was used in the system.
%See Table~\ref{somenumbers}.  This is a floating table environment.
%\LaTeX\ will try to put it close to where it was typeset but will not
%allow the table to be split.


%%SAMPLE TABLE WITH SHIT AND STUFF
%%\begin{table}[tbh]
%%\caption{A sample Table ... some numbers. \label{somenumbers}}
%%\begin{center}
%%\begin{tabular}{|r|l|}
  %%\hline
  %%7C0 & hexadecimal \\
  %%3700 & octal \\ \cline{2-2}
  %%11111000000 & binary \\
  %%\hline \hline
  %%1984 & decimal \\
  %%\hline
%%\end{tabular}
%%\end{center}
%%\end{table}

