% !TEX root = SystemTemplate.tex

%%GOT IT
\chapter{Design  and Implementation}

The design of the cluster is going change as we test different configurations to determine which is capable of producing the most gigaflops, and as we test different connection methods as we design our custom data transfer protocol. This section will outline the different designs we have created, how they were implemented, and our plans for implementing future designs going forward.
 

%%GOT IT
\section{Cluster Configure}

The first design we implemented was a star topology, with each device of the eight devices connected over Ethernet to a central switch. Each device was configured to be on the same network and capable of communicating over the switch via IP addresses. We configured the /home directory of our head node, Snow White, to be an NFS export that the seven other nodes would mount in their /home directory. 

%%GOT IT
\subsection{Technologies  Used}
We used several Linux system configuration tools to implement the cluster. We used the files
\begin{itemize}
	\item etc/network/interfaces
	\item etc/exports
	\item etc/hostname
	\item etc/hosts
	\item etc/fstab
\end{itemize}
for several different configuration setting. We also used a few packages availible from the defual debian repositories.
\begin{itemize}
	\item nfs-kernel-server
	\item nfs-common
	\item mpi-dev
\end{itemize}

Finally, we used SSH tools that are installed by default on Ubuntu 15.

%%GOT IT
\subsection{Component  Overview}
The features implented by this configuration were:

\begin{itemize}
	\item All devices recognizing the others over Ethernet.
	\item SSH without requiring a password.
	\item Mount home directory of Snow White on the dwarfs.
	\item Running MPI code on all devices.
\end{itemize}

\subsection{Phase Overview}

\subsection{ Architecture  Diagram}

%%GOT IT
\subsection{Design Details}
First, we made the devices able to communicate on the same network. We assigned each device a static IP address by editing the /etc/networking/interfaces file to incluce the IP address chosen for the device. All addresses were on the 192.168.1.1 network, and the last number was 11 through 18 for the eight devices.

Next, we set each device to be able to use our naming convention in place of an IP address for any purpose, such as by using "ssh sleepy" instead of "ssh 192.168.1.13". To do this, we changed the /etc/hostname file to replace "odroid" with the name we wanted, and added entries to /etc/hosts to include the IP address of the other seven decives. 

We then set the /home directory of Snow White to be an NFS share that could be mounted on the dwarfs. After nfs-kernel-server was installed on Snow White, we edited it's /etc/exports file to include /home as an export. The dwarfs then installed nfs-common and used the command "mount SnowWhite:/home /home" to mount the home directory of Snow White over their own.

To make this mount process automatic on boot, we edited the /etc/fstab file on each dwarf to make them mount Snow White's home directory as part of the boot process. This proved unsuccesful, however, on all devices except for on. As a work around, we wrote a Python tool that can be run from Snow White to mount it's home directory on the dwarfs.


\begin{lstlisting}
#!/bin/usr/python

import os

def Main():

        hosts = [ 'snow_white', 'dopey', 'sleepy', 'grumpy', 'doc', 'happy',
 				  'bashful', 'sneezy' ]

        for host in hosts:
                if host != 'snow_white':
                        cmd = "ssh odroid@" + host + " 'sudo mount -t nfs snow_white:/
							   home /home'"
                        os.system( cmd )

if __name__ == '__main__':
        Main()
\end{lstlisting}	
