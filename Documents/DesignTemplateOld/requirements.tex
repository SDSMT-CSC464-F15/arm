%old
% !TEX root = SystemTemplate.tex
\chapter{User Stories, Backlog and Requirements}
\section{Overview}


%The overview should take the form of an executive summary.  Give the reader a feel 
%for the purpose of the document, what is contained in the document, and an idea 
%of the purpose for the system or product. 

% The userstories 
%are provided by the stakeholders.  You will create he backlogs and the requirements, and %document here.  
%This chapter should contain 
%details about each of the requirements and how the requirements are or will be 
%satisfied in the design and implementation of the system.

%Below:   list, describe, and define the requirements in this chapter.  
%There could be any number of sub-sections to help provide the necessary level of 
%detail. 





\subsection{Scope}

%%******SAMANTHA.THIS SECTION NEEDED TO BE WRITTEN. IF YOU'RE NOT GOING TO WRITE IT, THEN AT LEAST COMMENT IT OUT
%What scope does this document cover?  This document would contain stakeholder information, 
%initial user stories, requirements, proof of concept results, and various research 
%task results. 



\subsection{Purpose of the System}
The system is used for research purposes and a proof of concept. 


\section{ Stakeholder Information}
One stakeholder is Dr. Christer Karlsson. If this project is successfully completed,
Dr. Karlsson plans to use the research and project results as a proof of concept. 


\subsection{Customer or End User (Product Owner)}
Dr. Christer Karlsson is the end user.  The end user might include 

%Who?  What role will they play in the project?  Will this person or group manage 
%and prioritize the product backlog?  Who will they interact with on the team to 
%drive product backlog priorities if not done directly? 

\subsection{Management or Instructor (Scrum Master)}
Dr. Karlsson manages this project and drives the meetings.


\subsection{Investors}
Dr. Karlsson is the investor on the project. His role is also the client.

\subsection{Developers --Testers}
Andrew Hoover, Samantha Kranstz, and Christine Sorensen developed and tested 
the cluster. 


\section{Business Need}
There is no buisness need. This project solely for research purposes.  


\subsection{System  Requirements}
The only system requirement would be the cluster must be made of single-board
computers.


\subsection{Network Requirements}
Create a new network for the cluster.


\subsection{Development Environment Requirements}
None. 


\subsection{Project  Management Methodology}
Oral progress reports are due on Tuesdays and Thursdays at one o'clock in the 
afternoon. These reports are given to Dr. Karlsson. 
 
\begin{itemize}
\item Trello is used to manage the backlog and status.
\item All parties have access to the sprint and product backlogs.
\item Six sprints will be completed this project
\item The sprint cycles are a couple weeks long.
\item No restrictions on source control.
\end{itemize}

\section{User Stories} 

\subsection{User Story \#1}
As a user, I want a cluster of at least 6 and no more than 12 single-board computers.
\subsubsection{User Story \#1 Breakdown}
The cluster will be made of ODROIDs, PcDuinos, or Raspberry Pi's, depending on which performs best in the benchmark tests.

\subsection{User Story \#2} 
As a user, I want the fasest, most efficient in both cost and operation cluster.
\subsubsection{User Story \#2 Breakdown}
Testing will be done on the single-board computers compared with prices to determine which will be best for the ARM cluster.

\subsection{User Story \#3} 
I want to the cluster to be at or below the maximum budget of \$1,200.00.
\subsubsection{User Story \#3 Breakdown}
The budget must include all components of the cluster: the computer boards, cost of power, switch, memory, cables, and power strips.

\subsection{User Story \#4}
I want to know which of the single-board computers is the fastest in GFlops/\$/Watt.
\subsubsection{User Story \#4 Breakdown}
Testing will take place on the ODROID, PcDuino, and Raspberry Pi to determine which is the fastest in this metric. 

\subsection{User Story \#5} 
I want a different communication mode beyond standard Ethernet.
\subsubsection{User Story \#5 Breakdown}
Utilize the other pins and ports to find an alternative form of communication.

\subsection{User Story \#6} 
Develop a message passing protocol for the communication.
\subsubsection{User Story \#6 Breakdown}
There is no message passing protocol for the other modes of communication. They must be developed and benchmarked.

\section{Research or Proof of Concept Results}
This project will be used as a proof of concept.
