% New Templete!!
% !TEX root = DesignDocument.tex


\chapter{Experimental Log}

For research projects one needs to keep a log of all research/lab activities.   

%% If you have multiple labs, you may want to break the labs into sections, check 
%% with the profession on format.
%% \section{Lab 1}

%%\begin{description}
%%\item [10/15/15]  Ran modified filter on data sets 1 - 6.  Results were ...
%%\item [10/17/15]  Changed tolerance on sensor and collected data.  These ...
%%\end{description}

\section{Benchmarking the Individual Computers}

\begin{description}
\item [9/17/15]  PcDuino isn't working according to Dr. Karlsson. The PcDuinos are about \$160 each, so chances were that the PcDuino wasn't going to be selected for the cluster. We will not put the PcDuino in consideration with our benchmarking.
\item [9/17/15]  PcDuino isn't working according to Dr. Karlsson. The PcDuinos are about \$160 each, so chances were that the PcDuino wasn't going to be selected for the cluster. We will not put the PcDuino in consideration with our benchmarking.
\item [9/22/15] We begin work on benchmarking the remaining candidates. The code we will be using to benchmark the two devices will test the addition, multiplication, division, trigonmetric function in single and double point precision of two massively large arrays filled with random numbers.
\item [9/29/15] OpenMP is added to the benchmark code so the program runs on all cores. Results are as follows: \newline

\begin{center}
\begin{tabular}{ | l || l | l | l | l | }
\hline
\multicolumn{5}
{ |c| }{ Length of Time (seconds) } \\
\hline
Device & Addition & Multiplication & Division & Sine \\
\hline
ODroid 4xU & 29.925 & 31.341 & 37.032 & 227.40 \\
\hline
Raspberry Pi 2B & 221.645 & 221.034 & 297.204 & 1468.63 \\
\hline
\end{tabular}
\end{center}

\item [9/30/15] The gigaflops are calculated. \newline
\begin{center}
\begin{tabular}{ | l || l | l | l | l | }
\hline
\multicolumn{5}
{ |c| }{ Gigaflops } \\
\hline
Device & Addition & Multiplication & Division & Sine \\
\hline
ODroid 4xU & 0.311 & 0.297 & 0.251 & 0.0410 \\
\hline
Raspberry Pi 2B & 0.0420 & 0.0421 & 0.0313 & 0.00634 \\
\hline
\end{tabular}
\end{center}

\item [10/1/15] The wattage is measured when the devices are running these operations. Using the wattage, the metric of GFlops/Dollar/Watt is calculated. \newline

\begin{center}
\begin{tabular}{ | l || l | l | l | l | }
\hline
\multicolumn{5}
{ |c| }{ Gigaflops per Dollar per Watts } \\
\hline
Device & Addition & Multiplication & Division & Sine \\
\hline
ODroid 4xU & 0.00028 & 0.000268 & 0.000226 & 0.0000369 \\
\hline
Raspberry Pi 2B & 0.0003 & 0.0003 & 0.000224 & 0.0000453 \\
\hline
\end{tabular}
\end{center}

\item [10/1/15] The results show that the Raspberry Pi and the ODroid perform nearly the same. The Raspberry Pi in our benchmarking proved the best. However, it is inconclusive as to which computer will be used.

\item [10/24/15] Decided to go with the ODroid. Performance and the number of ports outweighted the cost of the Raspberry Pi.

\end{description}

\section{Ethernet Benchmark}
\begin{description}
\item [10/25/15] Created network between a machine with gigabit ethernet port and ODroid. Installed n both the server machine and ODroid.

\begin{center}
\begin{tabular}{ | l || l | }
\hline
\multicolumn{2}
{ |c| }{ Ethernet Speed } \\
\hline
Device & Speed (Mbps) \\
\hline
ODroid XU4 & 615 - 625 \\
\hline
\end{tabular}
\end{center}

\end{description}

\section{Hardware Test}
\begin{description}
\item [11/03/15] All eight ODroids are functional. To test them, each device was connected to a router with internet access via ethernet. A monitor was connected through HDMI, and the mouse and keyboard were connected to both USB 3.0 ports to ensure they worked. The packages mpi-default-dev and openmpi-bin were installed. The test mpi code found this director and successfully compiled and executed. No issues found.
\end{description}

\section{Switch Benchmark}
\begin{description}
\item [11/03/15] With two ODroid devices attached to the switch via direct ethernet (no USB 3.0 to ethernet adapter), the connection speed was tested with iperf.

\begin{center}
\begin{tabular}{ | l || l | }
\hline
\multicolumn{2}
{ |c| }{ Switch Speed } \\
\hline
Device & Speed (Mbps) \\
\hline
ODroid XU4 & 775 - 800 \\
\hline
\end{tabular}
\end{center}

This is notably faster than the previous benchmark between one ODroid and a non-ODroid machine not using a switch.
\end{description}

\section{USB to Ethernet Benchmark}
\begin{description}
\item [11/03/15] Tested speed of USB 3.0 to ethernet adapter using iperf. Found slower than direct ethernet conection. Test speeds varied greatly. The USB ethernet adapted was faster acting as a server than a client.

\begin{center}
\begin{tabular}{ | l || l | }
\hline
\multicolumn{2}
{ |c| }{ USB 3.0 to Ethernet Adapter Speed } \\
\hline
Device & Speed (Mbps) \\
\hline
ODroid XU4 & 300 - 700 \\
\hline
\end{tabular}
\end{center}

In contrast, the ethernet connections were consistent reguardless of which device was the client or server.

\item [11/05/15] There isn't a found way to connect two devices over USB-ethernet to ethernet directly. When attached to the switch, the devices can communicate. If using the USB to ethernet adapter, they would be directly connected without the switch. Therefore, it was unable to directly connect devices. \\ \\
The drastically lower speed of the USB to ethernet adapters and the inability to directly connect the devices means that the devices are very unlikely to be useful for this project.
\end{description}

%skipping to sprint 4 right now
\section{USB to USB Communication}
\begin{description}
\item [1/26/16] Figuring out how to communicate through USB. In /etc/network/interface, changed USB's from static to dhcp. Then performed sudo ifup usb0.
Setting up the USB IP Addresses:
Go the the ODROID to change.


 	\$ wget http://tex.stackexchange.com
	\$ sudo modprob g_ether
	\$ dmesg
	\$ sudo ifconfig usb0 ipaddress 
	\$ config

\langle dog \rangle

Verify the USB IP address was changed.

Changed Doc, Happy, Bashful, and Sneezy's USB IP addresses:
Doc - 192.168.1.25
Happy - 192.168.1.26
Bashful - 192.168.1.27
Sneezy - 192.168.1.28

\item[2/1/16] Over USB 2.0, it is possible to use USB crossover cables to simulate Ethernet over USB. The same technology is possible for USB 3.0, but no operating system currently supports it, Linux or otherwise. As such, direct communication would have to be done in a different way than Ethernet, such as PyUSB. However, there is no found way to communicate from hsot node to another host node using PyUSB.
\end{description}