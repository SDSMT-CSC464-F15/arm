% NEW Template!!
% !TEX root = DesignDocument.tex

\chapter{User Stories,  Requirements, and Product Backlog}
\section{Overview}

%%The overview should take the form of an executive summary.  Give the reader a feel 
%%for the purpose of the document, what is contained in the document, and an idea 
%%of the purpose for the system or product. 

This document contains the full description of the ARM Cluster project, what was done, for what purpose, and the results encountered. 

%% The userstories 
%%are provided by the stakeholders.  You will create he backlogs and the requirements, and %%document here.  
%%This chapter should contain 
%%details about each of the requirements and how the requirements are or will be 
%%satisfied in the design and implementation of the system.

%%Below:   list, describe, and define the requirements in this chapter.  
%%There could be any number of sub-sections to help provide the necessary level of 
%%detail. 



\section{User Stories}
%%This section can really be seen as the guts of the document.  This section should 
%%be the result of discussions with the stakeholders with regard to the actual functional 
%%requirements of the software.  It is the user stories that will be used in the 
%%work breakdown structure to build tasks to fill the product backlog for implementation 
%%through the sprints.

%%This section should contain sub-sections to define and potentially provide a breakdown 
%%of larger user stories into smaller user stories.   Each component must have a test identified, 
%%meaning you need to know how you plan to test it.  If a requirement is not testable, then 
%%some justification needs to be made on why the requirement has been included.  
%% The results of the tests should go in the testing chapter. 

%% This section needs to be elaborated more.
%% -Christine

\subsection{User Story \#1}
As a user, I want a cluster of at least 6 and no more than 12 single-board computers.
\subsubsection{User Story \#1 Breakdown}
The cluster will be made of ODROIDs, PcDuinos, or Raspberry Pi's, depending on which performs best in the benchmark tests.

\subsection{User Story \#2} 
As a user, I want the fasest, most efficient in both cost and operation cluster.
\subsubsection{User Story \#2 Breakdown}
Testing will be done on the single-board computers compared with prices to determine which will be best for the ARM cluster.

\subsection{User Story \#3} 
I want to the cluster to be at or below the maximum budget of \$1,200.00.
\subsubsection{User Story \#3 Breakdown}
The budget must include all components of the cluster: the computer boards, cost of power, switch, memory, cables, and power strips.

\subsection{User Story \#4}
I want to know which of the single-board computers is the fastest in GFlops/\$/Watt.
\subsubsection{User Story \#4 Breakdown}
Testing will take place on the ODROID, PcDuino, and Raspberry Pi to determine which is the fastest in this metric. 

\subsection{User Story \#5} 
I want a different communication mode beyond standard Ethernet.
\subsubsection{User Story \#5 Breakdown}
Utilize the other pins and ports to find an alternative form of communication.

\subsection{User Story \#6} 
Develop a message passing protocol for the communication.
\subsubsection{User Story \#6 Breakdown}
There is no message passing protocol for the other modes of communication. They must be developed and benchmarked.

\section{Requirements and Design Constraints}
%%Use this section to discuss what requirements exist that deal with meeting the 
%%business need.  These requirements might equate to design constraints which can 
%%take the form of system, network, and/or user constraints.  Examples:  Windows 
%%Server only, iOS only, slow network constraints, or no offline, local storage capabilities.

The Product Owner stated that the cluster must be made using either PCDuinos, ODROID XU4s or Raspberry Pi 2Bs. These are likely to be  the most effective devices for our needs anyway, as they are the most affordable and powerful single board computers availible.

\subsection{System  Requirements}
%%What are they?  How will they impact the potential design?  Are there alternatives? 


\subsection{Network Requirements}
%%What are they?
The Network Requirements are largely left to be decided during the course of the project. 


\subsection{Development Environment Requirements}
%%What are they?  Is the system supposed to be cross-platform? 

\subsection{Project  Management Methodology}
%%The stakeholders might restrict how the project implementation will be managed. 
%% There may be constraints on when design meetings will take place.  There might 
%%be restrictions on how often progress reports need to be provided and to whom. 

%% This was where we had this in the old templete
%% If you look below, you'll see that it should be there maybe??
%% P.S. Samantha wrote this. Redo.
%% -Christine
Oral progress reports are due on Tuesdays and Thursdays at one o'clock in the 
afternoon. These reports are given to Dr. Karlsson. 
 
\begin{itemize}
\item Trello is used to manage the backlog and status.
\item All parties have access to the sprint and product backlogs.
\item Six sprints will be completed this project
\item The sprint cycles are a couple weeks long.
\item No restrictions on source control.
\end{itemize}


\section{Specifications}
%%Any specifications that need to be understood?  Put it here.  

\section{Product Backlog}
%%The full product backlog should go here.  The sprint backlogs are located in the project chapter.

 
%%\begin{itemize}
%%\item What system will be used to keep track of the backlogs and sprint status?
%%\item Will all parties have access to the Sprint and Product Backlogs?
%%\item How many Sprints will encompass this particular project?
%%\item How long are the Sprint Cycles?
%%\item Are there restrictions on source control? 
%%\end{itemize}


\section{Research or Proof of Concept Results}
%%This section is reserved for the discussion centered on any research that needed 
%%to take place before full system design.  The research efforts may have led to 
%%the need to actually provide a proof of concept for approval by the stakeholders. 
%% The proof of concept might even go to the extent of a user interface design or 
%%mockups. 

%% In Samantha's words: This project will be used as a proof of concept.


\section{Supporting Material}

%%This document might contain references or supporting material which should be documented 
%%and discussed  either here if appropriate or more often in the appendices at the end.  This material may have been provided by the stakeholders  
%%or it may be material garnered from research tasks.

