% NEW Template!!
% !TEX root = DesignDocument.tex

\chapter{User Stories,  Requirements, and Product Backlog}
\section{Overview}

%%The overview should take the form of an executive summary.  Give the reader a feel 
%%for the purpose of the document, what is contained in the document, and an idea 
%%of the purpose for the system or product. 

This document contains the full description of the ARM Cluster project, what was done, for what purpose, and the results encountered. 

%% The userstories 
%%are provided by the stakeholders.  You will create he backlogs and the requirements, and %%document here.  
%%This chapter should contain 
%%details about each of the requirements and how the requirements are or will be 
%%satisfied in the design and implementation of the system.

%%Below:   list, describe, and define the requirements in this chapter.  
%%There could be any number of sub-sections to help provide the necessary level of 
%%detail. 



\section{User Stories}
%%This section can really be seen as the guts of the document.  This section should 
%%be the result of discussions with the stakeholders with regard to the actual functional 
%%requirements of the software.  It is the user stories that will be used in the 
%%work breakdown structure to build tasks to fill the product backlog for implementation 
%%through the sprints.

%%This section should contain sub-sections to define and potentially provide a breakdown 
%%of larger user stories into smaller user stories.   Each component must have a test identified, 
%%meaning you need to know how you plan to test it.  If a requirement is not testable, then 
%%some justification needs to be made on why the requirement has been included.  
%% The results of the tests should go in the testing chapter. 

%% This section needs to be elaborated more.
%% -Christine

\subsection{User Story \#1}
As a user, I want a cluster of at least 6 and no more than 12 single-board computers.
\subsubsection{User Story \#1 Breakdown}
The cluster will be made of ODROIDs, PcDuinos, or Raspberry Pi's, depending on which performs best in the benchmark tests.

\subsection{User Story \#2} 
As a user, I want the fasest, most efficient in both cost and operation cluster.
\subsubsection{User Story \#2 Breakdown}
Testing will be done on the single-board computers compared with prices to determine which will be best for the ARM cluster.

\subsection{User Story \#3} 
I want to the cluster to be at or below the maximum budget of \$1,200.00.
\subsubsection{User Story \#3 Breakdown}
The budget must include all components of the cluster: the computer boards, cost of power, switch, memory, cables, and power strips.

\subsection{User Story \#4}
I want to know which of the single-board computers is the fastest in GFlops/\$/Watt.
\subsubsection{User Story \#4 Breakdown}
Testing will take place on the ODROID, PcDuino, and Raspberry Pi to determine which is the fastest in this metric. 

\subsection{User Story \#5} 
I want a different communication mode beyond standard Ethernet.
\subsubsection{User Story \#5 Breakdown}
Utilize the other pins and ports to find an alternative form of communication.

\subsection{User Story \#6} 
Develop a message passing protocol for the communication.
\subsubsection{User Story \#6 Breakdown}
There is no message passing protocol for the other modes of communication. They must be developed and benchmarked.

\section{Requirements and Design Constraints}
%%Use this section to discuss what requirements exist that deal with meeting the 
%%business need.  These requirements might equate to design constraints which can 
%%take the form of system, network, and/or user constraints.  Examples:  Windows 
%%Server only, iOS only, slow network constraints, or no offline, local storage capabilities.

The Product Owner stated that the cluster must be made using either PCDuinos, ODROID XU4s or Raspberry Pi 2Bs. These are likely to be  the most effective devices for our needs anyway, as they are the most affordable and powerful single board computers availible.


\subsection{Network Requirements}
%%What are they?
To operate the cluster, no external network will be required. The project will entail creating it's own internal network, and as such no access to the internet is necessary.

\subsection{Development Environment Requirements}
%%What are they?  Is the system supposed to be cross-platform?

All the research will be specific to the ARM architecture. The created code and packages will only be desgined with that in mind.

\subsection{Project  Management Methodology}
%%The stakeholders might restrict how the project implementation will be managed. 
%% There may be constraints on when design meetings will take place.  There might 
%%be restrictions on how often progress reports need to be provided and to whom. 

The project will be managed using Github as a repositoty to hold all source code, log files, and documentation. The Design Meeting with Dr. Karlsson will be on Tuesday and Thursday at 1:00pm in his office. The meetings will be to discuss any progress, the goals going forward, and obstacles being encountered. 

\section{Product Backlog}
%%The full product backlog should go here.  The sprint backlogs are located in the project chapter.

The main backlog for our project is as follows.

\begin{itemize}
	\item Decide on a computer based on the results of the benchmarking
	\item Calculate prices on supplies and computers while maintaining below the budget
	\item Ordering said supplies and computers
	\item Build the cluster to perform floating-point operations
	\item Benchmark the cluster
	\item Experiment with different connections
	\item Create a new mode of communication
	\item Design documentation
	\item Research symposium
\end{itemize}

The sprint status and backlog progress will be tracked on Trello and log files in our Github repository. As we accomplish goals for our project, we will move over Trello cards to the complete section, and describe our progress in log files that will be be uploaded in to our Documents section of our Github repository.

This system means that since Dr. Jeff McGough and Dr. Chrisster Karlsson, the Senior Design professor and our Product Owner, have access to our Github repository they will be able to see our backlog progress.

The work will be completed over six total sprints. They will each have their own backlog, being a subsection of the total backlog. Work not completed in any particular sprint will be carried over to the next sprint to be completed then.

Each sprint will last for three weeks, with the start and end times listed. \\

Sprint 1: 9/14/15 - 10/2/15 \\
Sprint 2: 10/12/15 - 10/30/15 \\
Sprint 3: 11/9/15 - 11/27/15 \\
Sprint 4: 1/18/16 - 2/5/16 \\
Sprint 5: 2/15/16 - 3/4/16 \\
Sprint 6: 3/21/16 - 4/15/16 \\

The time between the sprints will be used to complete sprint reports, have meetings with the team members and Product Owner as to the successes and failures of the sprints, and plan for what needs to be accomplished for the next sprints.


%%\begin{itemize}
%%\item What system will be used to keep track of the backlogs and sprint status?
%%\item Will all parties have access to the Sprint and Product Backlogs?
%%\item How many Sprints will encompass this particular project?
%%\item How long are the Sprint Cycles?
%%\item Are there restrictions on source control? 
%%\end{itemize}


\section{Research or Proof of Concept Results}
%%This section is reserved for the discussion centered on any research that needed 
%%to take place before full system design.  The research efforts may have led to 
%%the need to actually provide a proof of concept for approval by the stakeholders. 
%% The proof of concept might even go to the extent of a user interface design or 
%%mockups. 

%% In Samantha's words: This project will be used as a proof of concept.

This project will encompess a large amount of reseach of a wide variety of information, detailed elsewhere in this document. However, due to the nature of the project, most of the research will occurr as a part of the system design, not before hand. As ssuch, the results of said research will appear elsewhere in this document.

