% New template!
% !TEX root = DesignDocument.tex


\chapter{Project Overview}
%%This section provides some housekeeping type of information with regard to the 
%%team, project, environment, etc. 



\section{Team Member's Roles}
%Describe who was involved and what role(s) were played. 
\begin{itemize}
	\item Andrew Hoover - hardware/testing/software 
	\item Christine Sorensen - team lead/parallel programmer/software/documentation
\end{itemize}

\section{Project  Management Approach}
Project will be split into six sprints, each lasting two weeks. Items from the backlog are organized and assigned into these sprints. \newline \newline Product backlog is located on the Trello board. Documents and source code is located on Github. \newline \newline Formal meetings take place twice a week on Tuesdays and Thursdays at 1:00pm in advisor's, Dr. C. Karlsson's, office. Casual meetings are planned as needed.


\section{ Stakeholder Information}

%%This section would provide the basic description of all of the stakeholders for 
%%the project.  Who has an interest in the successful and/or unsuccessful completion 
%%of this project? 

The stakeholders include Dr. Christer Karlsson, as he is the sponser for the project and will keep the cluster after it is built. 


\subsection{Customer or End User (Product Owner)}
%%Who?  What role will they play in the project?  Will this person or group manage 
%%and prioritize the product backlog?  Who will they interact with on the team to 
%%drive product backlog priorities if not done directly? 

The Product Owner is also Dr. Christer Karlsson. He will provide the team with the product backlog, prioritize the goals for the project, and provide the hardware once the team decides what is needed. 

\subsection{Management or Instructor (Scrum Master)}
%%Who?  What role will they play in the project?  Will the Scrum Master drive the 
%%Sprint Meetings? 

Christine Sorensen will be the Scrum Master. She will be in charge of driving the Spring Meetings and leading the project.

\subsection{Investors}
%%Are there any?  Who?  What role will they play? 

Dr. Christer Karlsson will be the investor, as he will provide the money needed to build the cluster.

\subsection{Developers --Testers}
%%Who?  Is there a defined project manager, developer, tester, designer, architect, 
%%etc.? 

All members for the project will be responsible for a variety of roles. Aside from Christine Sorensen being the Scrum Master, the members of the team will share the workload.

\section{Budget}
%%Describe the budget for the project including gifted equipment and salaries for 
%%people on the project.

The total budget for this project is \$1,200. This includes all spending that will need to be done for the project. 

\section{Intellectual Property and Licensing}
%%Describe the IP ownership and issues surrounding IP.

All IP for the project will be owned by the school.

\section{Sprint  Overview}
%%If the system will be implemented in phases, describe those phases/sub-phases (design, 
%%implementation, testing, delivery) and the various milestones in this section. 
%%This section should also contain a correlation between the phases of development 
%%and the associated versioning of the system, i.e. major version, minor version, 
%%revision. 

%%All of the Agile decisions are listed here.  For example, how do you order your backlog?   
%%Did you use planning poker?

The project work will be divided in to Sprints with set backlogs. Each Sprint will contain it's own set of goals and results. The backlog will be ordered according to what the Product Owner Dr. Chirster Karlssson dictates. 

\section{Terminology and Acronyms}
%%*Provide a list of terms used in the document that warrant definition.  Consider 
%%industry or domain specific terms and acronyms as well as system specific. 
\begin{itemize}
	\item ARM - advanced RISC machine, a family of RISC archiectectures for computer processors.
	\item RISC - reduced instruction set computing, CPU design strategy based on simplified instruction set.
	\item Benchmarking - running tests in order to assess the perfomance of the computers.
	\item iperf - tool in standard Debian repositories to test network speed.
	\item LINAPCK - linear algebra equataions package, used by us for benchmarking.
	\item ATLAS - Automatically Tuned Linear Algebra Sofware, used by LINPACK.
	\item MPI - Message Passing Interface, used for paralled computing.	
	\item sudo - allows the user to run programs with the permissions of the super user.
	\item vim - a text editor in Unix-like operating systems used in most of the logs and examples.
	\item SSH - secure shell, a network protocol to allow remote login from other network services.
\end{itemize}

\section{Sprint Schedule}
%%The sprint schedule.  Can be tables or graphs.   This can be a list of dates with the visual 
%%representation given below.

Sprint 1: 9/14/15 - 10/2/15 \\
Sprint 2: 10/12/15 - 10/30/15 \\
Sprint 3: 11/9/15 - 11/27/15 \\
Sprint 4: 1/18/16 - 2/5/16 \\
Sprint 5: 2/15/16 - 3/4/16 \\
Sprint 6: 3/21/16 - 4/15/16 \\

%%\section{Timeline}
%%Gantt chart or other type of visual representation of the project timeline.


\section{Backlogs}
%%Place the sprint backlogs here.    The product backlog will be in the chapter with the user 
%%stories.

\subsection*{Sprint 1 Backlog}
\begin{itemize}
	\item Decide on a computer based on the results of the benchmarking
	\item Calculate prices on supplies and computers while maintaining below the budget
	\item Ordering said supplies and computers
	\item Build the cluster to perform floating-point operations
	\item Benchmark the cluster
	\item Experiment with different connections
	\item Create a new mode of communication
\end{itemize}

\subsection*{Sprint 2 Backlog}
\begin{itemize}
	\item Build the cluster to perform floating-point operations
	\item Benchmark the cluster
	\item Experiment with different connections
	\item Create a new mode of communication
\end{itemize}

\subsection*{Sprint 3 Backlog}
\begin{itemize}
	\item Research new connection methods
	\item Benchmark the cluster
	\item Experiment with different topologies
	\item Create a new mode of communication
	\item Design documentation
	\item Research symposium
	\begin{itemize}
		\item Complete abstract
	\end{itemize}
	\item Design Fair
\end{itemize}

\subsection*{Sprint 4 Backlog}
\begin{itemize}
	\item MICS presentation.
	\item SDSMT Research Symnposium. 
	\item Design Documentation.
	\item Design Fair.
\end{itemize}

\subsection*{Sprint 5 Backlog}
\begin{itemize}
	\item MICS presentation.
	\item SDSMT Research Symnposium.
	\item Completed hypercube cluster.
	\item Conglomerate data results.
	\item Design Documentation.
	\item Design Fair.
\end{itemize}

\section{Source  Control}
%%Which source control system is/was used?  How was it setup?  How does a developer 
%%connect to it? 

Github will be used for our source control. Our repository was created by Dr. Jeff McGough for our use during the project. The repo is set to be public, so everyone has read access. Both team members, Dr. McGough and the Product Owner Dr. Christer Karlsson have write access.

\section{Build  Environment}
%%How are the packages built?  Are there build scripts?

The build environment for all the project was a version of Ubuntu 15 specifically made by HardKernel for use on the ODROID XU4. All code and processes work on the provided kernel.
